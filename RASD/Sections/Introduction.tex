\section{Purpose}
\ac{CKB} is a new platform that helps students improve their software development skills by training with peers on code katas . Educators use the platform to challenge students by creating code kata battles in which teams of students can compete against each other, thus proving (and improving) their skills.\newline
A code kata battle is essentially a programming exercise in a programming language of choice (e.g., Java, Python). The exercise includes a brief textual description and a software project with build automation scripts (e.g., a Gradle project in case of Java sources) that contains a set of test cases that the program must pass, but without the program implementation. Students are asked to complete the project with their code. In particular, groups of students participating in a battle are expected to follow a test-first approach and develop a solution that passes the required tests. Groups deliver their solution to the platform (by the end of the battle). At the end of the battle, the platform assigns scores to groups to create a competition rank.
\subsection{Goals}
The \ac{CKB} system is thought, designed and proposed to two types of users: Students and Educators.\newline
The firsts, will be able to join the platform to test their coding skills in Tournaments, a series of code battles, where students will be able to participate in groups.\newline
Educators instead will be able to create and manage tournaments and decide whether or not a manual score in a battle is needed or not.\newline
Below the list of goals of the \ac{CKB} platform.\newline

\begin{center}
    \begin{longtable}{ |l|p{0.9\linewidth}| }
        \hline
        \textbf{ID} & \textbf{Description}\\
        \hline
        G1 & An Educator can manage a tournament. \\
        \hline
        G2 & An educator can create battles inside of a tournament in which he is involved. \\
        \hline
        G3 & Students can participate in tournaments created by an educator\\
        \hline
        G4 & Students can participate and compete in battles created by an educator, alone or in groups. \\
        \hline
        G5 & Students are scored based on their performance in battles. \\
        \hline
        G6 & The platform allows students and educators to compare the performance of students. \\
        \hline
        \caption{The goals.}
    \end{longtable}
\end{center}


\section{Scope}
\subsection{World phenomena}
\begin{center}
    \begin{longtable}{ |l|p{0.9\linewidth}| }
        \hline
        \textbf{ID} & \textbf{Description}\\
        \hline
        W1 & An educator wants to create a tournament to evaluate students performance. \\
        \hline
        W2 & Students fork the created repository for a battle on Github. \\
        \hline
        W3 & An educator creates the battle assignment. \\
        \hline
        W4 & Students write source code for the code kata battle. \\
        \hline
        W5 & Students create commits on Github. \\
        \hline
        W6 & Students decide to join a battle. \\
        \hline
        W7 & Students create groups for battles. \\
        \hline
        W8 & Students setup an automated workflow for the forked repository on Github. \\
        \hline
        W9 & Students decide to join a tournament. \\
        \hline
        W10 & Educator decides to close tournament. \\
        \hline
        \caption{World Phenomenas.}
    \end{longtable}
\end{center}
\subsection{Shared phenomena}

\begin{center}
    \begin{longtable}{ |l|p{0.5\linewidth}|l|l| }
        \hline
        \textbf{ID} & \textbf{Description} & \textbf{Controller} & \textbf{Observer}\\
        \hline
        SP1 & An educator fills out a tournament creation form. & Educator & \ac{CKB}\\
        \hline
        SP2 & An educator uploads the details of a code kata battle(the assignment, the rules, the tests). & Educator & \ac{CKB} \\
        \hline
        SP3 & A group(maybe singleton) joins a battle respecting the rules regarding the min and max group size. & Student & \ac{CKB}\\
        \hline
        SP4 & An educator logs into the platform. & Educator & \ac{CKB} \\
        \hline
        SP5 & A student logs into the platform. & Student & \ac{CKB}\\
        \hline
        SP6 &  The system requires additional manual evalution by an educator for a battle(if required by the rules). & Educator & \ac{CKB}\\
        \hline
        SP7 & The educator inserts an additional manual score for a battle. & Educator & \ac{CKB}\\
        \hline
        SP8 & A student invites other students to a group to participate to a battle. & Student & \ac{CKB}\\
        \hline
        SP9 & Github on commit notifies the code kata platform. & GitHub & \ac{CKB}\\
        \hline
        SP10 & An educator registers an account on the platform. & Educator & \ac{CKB}\\
        \hline
        SP11 & A student registers an account on the platform. & Student & \ac{CKB}\\
        \hline
        SP12 & A student subscribes to a tournament. & Student & \ac{CKB}\\
        \hline
        SP13 & Students and educators look at the rank for a battle they are involved in. & User & \ac{CKB} \\
        \hline
        SP14 & Students and educator look at the rank for a tournament. & User & \ac{CKB}\\
        \hline
        SP15 & Educator closes tournament. & Educator & \ac{CKB}\\
        \hline
        SP16 & User looks at list of available tournament. & User & \ac{CKB}\\
        \hline
        SP17 &  A student accepts and invitet o a group to partecipate to a battle. & Student & \ac{CKB}\\
        \hline
        SP18 & The platform notifies all students when a tournament is created. & \ac{CKB} & Student\\
        \hline
        SP19 & The platfotm notifies all students subscribed to a tournament of new upcoming battles.& \ac{CKB} & Student\\
        \hline
        SP20 & The platform notifies the final score to all students subscribed to a battle, when that battle ends.& \ac{CKB} & Student\\
        \hline
        SP21 & When the platform is notifies about a commit, it pulls from the committed repository to start the mandatory analysis.& \ac{CKB} & Github\\
        \hline
        SP22 & The platform creates the github repository for a battle.& \ac{CKB} & Github\\
        \hline
        SP23 & The platform sends links to the created repository for the battle to all students who are subscribed to the battle.& \ac{CKB} & Student\\
        \hline
        \caption{Shared Phenomenas.}
    \end{longtable}
\end{center}

\section{Definitions, Acronyms, Abbreviations}

\begin{table}[H]
    \begin{center}
        \begin{tabular}{|l|l|}
            \hline
            \textbf{Acronym} & \textbf{Definition}\\
            \hline
            CKB & CodeKataBattle\\
            \hline
            CK & Code Kata\\
            \hline
        \end{tabular}
        \caption{Acronyms used in the document}
    \end{center}
\end{table}

\textbf{Educator involved in a tournament}: These are the educator who created the tournament and the collaboratorating educators who were granted ability to create a battle. \newline
\textbf{Battles in tournament context}: These are the battles created within a tournament by an involved educator, all students subscribed to said tournament can join them.

\section{Document Structure}
The document is divided in six sections. \newline
\textbf{The first section} introduces the goals of the project and shared phenomena with also a list of definitions useful to understand the problem.\newline
\textbf{The second section} provides a more accurate description of the problem, describing the details of domain and scenarios.\newline
\textbf{The third section} focuses on specific requirements and provides an analysis on interface requirements, functional requirements and so on.\newline
\textbf{The fourth section} provides a formal analysis using Alloy, crucial to prove the correctness of the model described.\newline
\textbf{The fifth one} reports the effort spent by each group member in the redaction of this document and the last section is simply a list of bibliography references.