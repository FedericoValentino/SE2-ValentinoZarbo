\section{Purpose}
\ac{CKB} is a new platform that helps students improve their software development skills by training with peers on code katas . Educators use the platform to challenge students by creating code kata battles in which teams of students can compete against each other, thus proving (and improving) their skills.\newline
A code kata battle is essentially a programming exercise in a programming language of choice (e.g., Java, Python). The exercise includes a brief textual description and a software project with build automation scripts (e.g., a Gradle project in case of Java sources) that contains a set of test cases that the program must pass, but without the program implementation. Students are asked to complete the project with their code. In particular, groups of students participating in a battle are expected to follow a test-first approach and develop a solution that passes the required tests. Groups deliver their solution to the platform (by the end of the battle). At the end of the battle, the platform assigns scores to groups to create a competition rank.

\section{Scope}
In this document we describe how we went about testing the implementation of the group formed by Matteo Panarotto, Tommaso Tognoli and Oliver Mosgaard Stege. 
\begin{itemize}
    \item Repository URL:
        \href{https://github.com/Smrevilo/StegeTognoliPanarotto}{\textunderscore{Link}}
\end{itemize}


\section{Definitions, Acronyms, Abbreviations}

\begin{table}[H]
    \begin{center}
        \begin{tabular}{|l|l|}
            \hline
            \textbf{Acronym} & \textbf{Definition}\\
            \hline
            CKB & CodeKataBattle\\
            \hline
            CK & Code Kata\\
            \hline
            UI & User Interface\\
            \hline
        \end{tabular}
        \caption{Acronyms used in the document}
    \end{center}
\end{table}

\section{Revision History}

\section{Reference Documents}

\begin{itemize}
    \item RASD Document for \ac{CKB} from group
    \item DD Document for \ac{CKB} from group
    \item IT Document for \ac{CKB} from group
\end{itemize}

\section{Document Structure}
This document is structured as follows:
\begin{itemize}
    \item Chapter 1 contains the introduction to the document;
    \item Chapter 2 goes about the installation process;
    \item Chapter 3 explains how the testing went;
    \item Chapter 4 contains any additional points worth raising;
    \item Chapter 5 contains the effort spent by each group member;
    \item Chapter 6 lists tools and references used;
\end{itemize}
